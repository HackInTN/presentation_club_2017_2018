\documentclass{beamer}

\usepackage[T1]{fontenc}
\usepackage[utf8]{inputenc}
\usepackage[english]{babel}
\usepackage{lmodern}
% \usepackage{forloop}
\usepackage{multicol}
\usepackage{animate}
\usepackage{default}
\usepackage{listing}

\usepackage[list=true]{subcaption}
\captionsetup{compatibility=false}
\usepackage{etex}


\usepackage{tikz}
% \usepackage{polyglossia}
% \usepackage{listings}
% \usepackage{ulem}
% \usepackage{multicol}

% \setbeamertemplate{navigation symbols}{}
% \setbeamertemplate{sidebar right}{}
% \setbeamertemplate{footline}{\hfill\insertframenumber{} | \inserttotalframenumber}


% \newcommand{\myline}{\par
%     \kern0.5pt
%     \hrule height 0.8pt
%     \kern0.5pt
% }

\usecolortheme{cormorant}
\useoutertheme{infolines}

\begin{document}


\title[Présentation]{H4ck 1n TN}
\subtitle{Présentation du club}
\author[H4ck1nTN]{
Riwan Dad (Respo Présentations)
\and \\
Julien Bataille (Respo CTF)
\and \\
Thérèse Thiéry (Secrétaire)
\and \\
Maël Houbre (Trésorier)
\and \\
Lucas Vignali (Vice-Président)
\and \\
Romain Karpinski (Président)
}

\institute[HiT]{Ceten -- TELECOM Nancy}
\date{\today}
\logo{\includegraphics[width=1.3cm]{logo.png}}


% Inscription au ceten obligatoire (12h à 12h20 et de 13h40 à 14h devant le BDE jusqu'à vendredi)

\begin{frame}{Objectifs du club}
	\begin{itemize}
		\item Gagner en connaissance sur le thème de la sécurité (mais pas que!)
		\item Apprendre des choses qu'on ne voit pas forcément à Télécom
		\item Partager ses connaissances
		\item S'amuser :)
	\end{itemize}
	
\end{frame}

\begin{frame}{Les activités}
	\begin{itemize}
		\item Présentation d'outils, de failles...
		\item Des évènements (Confs, Ateliers, Session rootme (CTFs), ...)
		\item Venez nous proposer des activités !
	\end{itemize}
\end{frame}

\begin{frame}{Informations utiles}
	\begin{itemize}
		\item Site du club: https://hackintn.telecomnancy.net (pas dispo pour le moment)
		\item Github du club: github.com/HackInTN
		\item Discord : https://discord.gg/N99c9UK
	\end{itemize}
\end{frame}

\begin{frame} {Le GreHack}
	\begin{itemize}
		\item Quoi : Conférences hacking \& workshops \& CTF 
		\item Quand : Vendredi 18 -> Samedi 19  novembre
		\item Où : Ensimag, Grenoble 
		\item plus d'info sur grehack.fr
	\end{itemize}
\end{frame}

\begin{frame} {Le GreHack}
	\begin{itemize}
		\item Tout le monde ne pourra pas venir ! (2-3 places)
		\item A gagner: Des goodies, de l'expérience, 5 points CIPA (représentation de l'école)
	\end{itemize}
\end{frame}

\begin{frame}{Sites d'entraînement}
	\begin{itemize}
		\item root-me.org (Site de challenge en tout genre\\
		Cracking, Web, Crypto, Réseau...)
		\item overthewire.org (Site de challenges orientés systeme)
		\item github.com/ctfs (write-ups des CTF passés)
        \item www.cybrary.it		
		\item canyouhack.it
	\end{itemize}
\end{frame}

\begin{frame}{Distributions}
	\begin{itemize}
    	\item La plupart des outils sont sous Linux (mais pas tous !)
		\item Ils existe des challenge spécifiques Windows
		\item Le club TekTN propose une install party à la fin du mois
	\end{itemize}
\end{frame}

\begin{frame}{Quelques outils}
	\begin{itemize}
		\item nmap
		\item wireshark, tcpdump, Dsniff
		\item john
		\item nikto, owasp zap
		\item aircrack-ng (wifite), btscanner/bluesnarfer
		\item Metasploit Framework
		\item SEToolkit
	\end{itemize}
\end{frame}

\begin{frame}{Quelques outils}
	\begin{itemize}
		\item strace, ltrace
		\item gdb, valgrind
		\item radare2, IDA (désassembleur/analyseur de code)
		\item binwalk
	\end{itemize}
\end{frame}


\begin{frame} 
	\begin{itemize}
    	\item Merci
		\item Questions ?
	\end{itemize}
\end{frame}


\end{document}
